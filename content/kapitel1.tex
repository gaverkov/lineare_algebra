\section{Mathematische Grundbegriffe}

\subsection{Aussagen}

\subsubsection{Aussage}

Eine Aussage ist ein Satz, der entweder wahr oder falsch ist. Etwa:
\begin{itemize}
	\item $ 2 < 1 $ (falsch)
	\item $ 2 = 1 $ (falsch)
	\item $ 2 > 1 $ (wahr)
\end{itemize}
\noindent Keine Aussagen:
\begin{itemize}
	\item Hallo!
	\item Was gibt's Neues?
\end{itemize}


\begin{bem}[Benennung von Aussagen]\ Bei der Kategorisierung von beweisbaren mathematischen Aussagen gibt es die folgenden Tendenzen: 
	\begin{itemize}
		\item Proposition: relativ einfach zu beweisen
		\item Theorem, Satz: bemerkenswert oder schwer zu beweisen
		\item Lemma, Hilfssatz:  Hilfsaussage, die  technisch  ist und beim Beweis eines Theorems eingesetzt wird. 
	\end{itemize}
\end{bem}


\subsubsection{Logische Verknüpfungen}

Seien $ A $ und $ B $ Aussagen. Dann definiert man anhand von $ A $ und $ B $ die folgenden Aussagen:
\begin{itemize}
	\item $ A \wedge B $ Konjunktion (\glqq und\grqq)
	\item $ A \vee B $ Disjunktion (\glqq oder\grqq)
	\item $ A \Rightarrow B $ Implikation
	\item $ A \Leftrightarrow B $ Äquivalenz
	\item $ A \:\dot{\vee}\: B $ ausschließende Disjunktion
	\item $ \neg A $, $ \bar{A} $ Negation (Verneinung)
\end{itemize}

\begin{bsp}\
	\begin{itemize}
		\item $ x,y \in \R, x = y \Rightarrow x^2 = y^2 $ (wahr)
		\item $ x,y \in \R, x^2 = y^2 \Rightarrow x = y $ (falsch)	
	\end{itemize}
\end{bsp}

\clearpage
\subsection{Mengen}

\subsubsection{Menge}

Eine Menge ist eine Ansammlung von Objekten. Diese Objekte nennt man Elemente der Menge. (intuitive Beschreibung, keine exakte Definition)

Eine Weise, Mengen zu definieren, ist durch die Auflistung ihrer Elemente. Dabei stehen die geschweiften Klammer für Mengen, die drei Punkte bedeuten \glqq usw\grqq.
\begin{itemize}
	\item $ \{1,2,5,7\} $
	\item $ \{1\} $
	\item $ \{1,\{2,5\},\{6\}\} $
	\item $ \{1,2,3,\ldots\} $
\end{itemize}
Ist $ x $ Element der Menge $ X $, so schreibt man $ x \in X $. Ist $ x $ nicht Element der Menge $X$, so schreibt man $ x \notin X $.

Seien $ A $ und $ B $ Mengen. Dann ist $ A $ genau dann eine Teilmenge von $ B $, wenn jedes Element von $ A $ auch Element von $ B $ ist ($ A \subseteq B \Leftrightarrow \forall\: x \in A : x \in B $).\footnote{In einigen mathematischen Quellen bezeichnet man die Inklusion als $ \subset $ und nicht als $ \subseteq $. Es gibt aber auch Quellen, in denen $ \subset $ die strikte Inklusion bezeichnet. Daher ziehe ich persönlich $ \subseteq $ vor.} Zwei Mengen $ A $ und $ B $ heißen genau dann gleich, wenn $ A \subseteq B $ und $ B \subseteq A $ gilt. $ A $ heißt genau dann echte Teilmenge einer Menge $ B $, wenn $ A \subseteq B $ und $ A \neq B $ erfüllt sind. Bezeichnung: $ A \varsubsetneq B $.

\subsubsection{Zahlenmengen}

\begin{itemize}
	\item[$ \N $] $ := \{ 1,2,3,\ldots \} $ die Menge der natürlichen Zahlen\footnote{Die Definition des Begriffs natürliche Zahl sowie die Bedeutung der jeweiligen Bezeichnung $\N$ für die Menge der natürlichen Zahlen sind abhängig von einer konkreten Quelle.  Manche Quellen definieren die Menge der natürlichen Zahlen als $ \set{0,1,2,\ldots} $, es ist mittlerweile sogar die ISO-Norm 80000-2, es gibt aber trotz der ISO-Norm sehr viele Quellen, in denen $0$ nicht in die Menge der natürlichen Zahl aufgenommen wird. Tendenziell wählt man  $\N = \{1,2,3,\ldots \}$ in der Analysis und $\N = \{0,1,2,\ldots\}$ in der theoretischen Informatik, Mengenlehre sowie der diskreten Mathematik.}
	\item[$ \N_0 $] $ := \{ 0,1,2,\ldots \} $
	\item[$ \Z $] $ := \{ 0,1,-1,2,-2,\ldots \} $ ganze Zahlen
	\item[$ \Q $] $ := \{ \frac{p}{q} : p \in \Z, q \in \N \} $ rationale Zahlen
	\item[$ \R $] reelle Zahlen
	\item[$ \C $] komplexe Zahlen
\end{itemize}

\noindent $ \Rightarrow \N \subseteq \N_0 \subseteq \Z \subseteq \Q \subseteq \R \subseteq \C $


\subsubsection{Definition durch eine Bedingung}

Eine weitere Weise, Mengen zu definieren, ist durch Bedingungen. Format:
\[
			 \{ \text{AUSDRUCK} \,  : \, \text{BEDINGUNG} \}
\]
Die Lesart für den Doppelpunkt ist ``sodass''  bzw.  ``mit der Bedingung''.

Zum Beispiel beschreibt die Formel
 \[
 	\{ k^2 : k \in \N, k \, \text{ungerade} \} 
 \] die Menge aller Quadrate von positiven ungeraden Zahlen. Eine andere Beschreibung für dieselbe Menge ist 
 \[
 	\{  (2i -i)^2  \, :\, i \in \N\}.
 \] 



\subsubsection{Die leere Menge}

Die leere Menge ist die Menge, die keine Elemente enthält. Bezeichnung: $ \emptyset $.

\subsubsection{Potenzmenge}

Sei $ X $ eine Menge. Dann ist die Potenzmenge von $ X $ die Menge aller Teilmengen von $ X $. Bezeichnung: $ 2^X $. Formal: $ 2^X := \{ A : A \subseteq X \} $. 

Anmerkung: Ist $ | X | = n \in \N_0$, so gilt $| 2^X | = 2^n $.


\subsubsection{Mengenoperationen}

Seien $ A,B $ Mengen. Dann heißt
\begin{itemize}
\item $ A \cap B := \{ x : (x \in A) \wedge (x \in B) \} $ Durchschnitt von $ A $ und $ B $
\item $ A \cup B := \{ x : (x \in A) \vee (x \in B) \} $ Vereinigung von $ A $ und $ B $
\item $ A \setminus B := \{ x : (x \in A) \wedge (x \notin B) \} $ Mengendifferenz von $ A $ und $ B $
\end{itemize}

\subsubsection{Disjunkte Mengen}

Seinen $ A,B $ Mengen. $ A $ und $ B $ heißen genau dann disjunkt, wenn $ A \cap B = \emptyset $.

\subsection{Tupel}

Für Objekte $ a,b $ kann man das \emph{geordnete Paar} $ (a,b) $ definieren. Für Objekte $ a,b,c,d $ definiert man die Gleichheit $ (a,b) = (c,d) $ durch $ a=c $ und $ b=d $. $ a $ heißt das erste Element des Paares $ (a,b) $ und $ b $ heißt das zweite Element.

Für Mengen $ X,Y $ definiert man das \emph{Kreuzprodukt} $ A \times B $ durch $ A \times B := \{ (x,y) : x \in X, y \in Y \} $. Analog definiert man geordnete Tupel und das Kreuzprodukt $ A \times B \times C $ von Mengen $ X,Y $ und $ Z $. Noch allgemeiner kann man für jedes $ n \in \N $ geordnete $ n $-Tupel $ (x_1,\ldots,x_n) $ einführen und das Kreuzprodukt $ X_1 \times \ldots \times X_n := \{ (x_1,\ldots,x_n) : x_1 \in X_1, \ldots , x_n \in X_n \} $ von Mengen $ X1,\ldots,X_n $.

Für eine Menge $ X $ führt man die Bezeichnung
\begin{equation*}
	X^n := \underbrace{X \times \ldots \times X}_{n \:\text{mal}} = \{ (x_1,\ldots,x_n) : x_1,\ldots,x_n \in X \}.
\end{equation*}
Das Element $ x_i $ mit $ i \in \is{1}{n} $ im $ n $-Tupel $ (x_1,\ldots,x_n) $ heißt die $ i $-te \emph{Komponente} des Tupels.

\begin{bsp} Geometrische Veranschaulichung einiger Tupel:
	\begin{itemize}
		\item $ [0,1] \times [0,2] $ ist Rechteck in $ \R^2 $\\
		$ \{ 0 \} \times [0,2] $ ist eine Kante dieses Rechtecks\\
		$ \{ 0,1 \} \times \{ 0,2 \} $ sind Eckpunkte dieses Rechtecks
		\item $ [0,1]^3 $ ist Würfel in $ \R^3 $\\
		$ [0,1]^2 \times \{ 0 \} $ ist eine Seitenfläche des Würfels\\
		$ [0,1]^2 \times \{ 1 \} $ ist gegenüberliegende Seitenfläche (Facette)\\
		$ \{ 0 \}^2 \times [0,1] $ ist eine Kante des Würfels
		\item $ [0,1]^4 $ ist ein 4-dimensionaler Würfel in $ \R^4 $
	\end{itemize}
\end{bsp}


\subsection{Abbildungen}

\subsubsection{Abbildung}

Seien $ X,Y $ Mengen. Eine Abbildung $ f $ von $ X $ nach $ Y $ ist eine Vorschrift, die jedem $ x \in X $ genau ein Element aus $ Y $ zuordnet. Dieses Element aus $ Y $ wird durch $ f(x) $ bezeichnet.
% 16.10.2014
Wenn $ f $ eine Abbildung von $ X $ nach $ Y $ ist, dann bezeichnet man das durch: $ f : X \to Y $. Die Menge $ X $ heißt der Definitionsbereich von $ f $, $ Y $ heißt der Wertebereich von $ f $.

%Beispiel:
%\begin{enumeration}
%\item Kekse
%\item Nüsse
%\item Riegel
%\end{enumeration}
%
%\begin{table}[h]
%	\begin{tabular}{c|c|c}
%		$ X $ & $ Y $ & $ Y $ \\
%		\hline
%		1 & Kekse & Kekse \\
%		2 & Kekse & Nüsse \\
%		3 & Kekse & Riegel
%	\end{tabular}
%\end{table}

\begin{itemize}
	\item $ f : \R \to \R,\\ f(x) := x^2 -2x + 7 \:\forall\: x \in \R $
	\item $ f : \R \setminus \{ 1 \} \to \R,\\ f(x) := \frac{1}{x-1} \:\forall\: x \in \R $
	\item $ \sign: \R \to \R = \{ 1,0,1 \} $
	\item $ f : 2^{\N} \to \N, f(A) := \min(A) \:\forall\: A \subseteq \N $, z.B. $ f( \{ 1,7,43 \} ) = 1 $
	\item $ f : \N \to 2^\N, f(k) := \is{1}{k} \:\forall\: k \in \N $
\end{itemize}

Zwei Abbildungen $ f,g : X \to Y $ heißen gleich, falls $ f(x) = g(x) $ für alle $ x \in X $ gilt. Durch $ Y^X $ bezeichnet man die Menge aller Abbildungen von $ X $ nach $ Y $.

\subsubsection{Bild und Urbild}

Seien $ X,Y,A,B $ Mengen mit $ A \subseteq X $ und $ B \subseteq Y $. Sei $ f : X \to Y $. Dann heißt $ f(A) := \{ f(x) : x \in A \} $ das Bild von $ A $ bzgl. $ f $ und $ f^{-1}(B) := \{ x \in X : f(x) \in B \} $ das Urbild von $ B $ bzgl. $ f $.

\begin{bsp}
	Sei $ f : \R \to \R $ mit $ f(x) := x^2 $ für alle $ x \in \R $.
	\begin{itemize}
		\item $ f( [1,2] ) = [1,4] $ $ \nearrow $ Abb. 1
		\item $ f^{-1}( [1,4] ) = [1,2] \cup [-2,-1] $ $ \nearrow $ Abb. 2
		\item
		$ \begin{aligned}[t]
			f^{-1}( [-7,8] ) &= \{ x \in \R : f(x) \in [-7,8] \} \\
			&= \{ x \in \R : -7 \leq f(x) \leq 8 \} \\
			&= \{ x \in \R : -7 \leq x^2 \leq 8 \} \\
			&= \{ x \in \R : x^2 \leq 8 \} \\
			&= \{ x \in \R : |x| \leq \sqrt{8} \} \\
			&= [-\sqrt{8},\sqrt{8}]
		\end{aligned} $
\end{itemize}
\end{bsp}

\begin{bem}[Intervalle]
	Seien $ a,b \in \R $ mit $ a \leq b $. Dann können Intervalle wie folgt definiert werden:
	\begin{align*}
		[a,b] &:= \{ x \in \R : a \leq x \leq b \}\\
		(a,b) &:= \{ x \in \R : a < x < b \}\\
		(a,b] &:= \{ x \in \R : a < x \leq b \}\\
		[a,b) &:= \{ x \in \R : a \leq x < b \}
	\end{align*}
\end{bem}

\subsubsection{Injektivität, Surjektivität und Bijektivität}

Seien $ X,Y $ Mengen und sei $ f : X \to Y $. Dann heißt $ f $:
\begin{itemize}
	\item injektiv, falls für alle $ x_1, x_2 \in X : x_1 \neq x_2 $ die Bedingung $ f(x_1) \neq f(x_2) $ gilt.
	\item surjektiv, falls für jedes $ y \in Y $ ein $ x \in X $ mit der Eigenschaft $ f(x) = y $ existiert.
	\item bijektiv, falls $ f $ injektiv und surjektiv ist.
\end{itemize}

\begin{bsp}
	Untersuche folgende Funktionen auf Bijektivität:
	\begin{itemize}
		\item $ f : \R \to \R, f(x) := x^2 $ für alle $ x \in \R $\\
		surjektiv ? nein, $ -1 \neq f(x) $ für alle $ x \in \R $\\
		injektiv ? nein, $ f(x) = f(-x) $ für alle $ x \in \R $
		\item $ f : \R \to [0,+\infty), f(x) := x^2 $ für alle $ x \in \R $\\
		surjektiv ? ja\\
		injektiv ? nein (analog)
		\item $ f : \R \setminus \{ 0 \} \to \R, f(x) = \frac{1}{x} $ für alle $ x \in \R $\\
		surjektiv ? nein, 0 wird nicht angenommen\\
		injektiv ? ja
		\item $ f : \R \to R, f(x) = 2x + 3 $ für alle $ x \in \R $\\
		bijektiv ? ja
	\end{itemize}
\end{bsp}

\subsubsection{Umkehrfunktion}

Seien $ X,Y $ Mengen und sei $ f : X \to Y $ bijektiv. Die Abbildung, die jedem $ y \in Y $ das eindeutige $ x \in X $ mit $ f(x) = y $ zuordnet, heißt die Umkehrabbildung von $ f $ und wird durch $ f^{-1} $ bezeichnet.

\begin{bsp}
	Die Umkehrung von $ f : \R \to \R : f(x) := 2x + 3 $ ist $ f^{-1}(x) = \frac{x-3}{2} $ $ ( x \in \R ) $.
\end{bsp}

\subsubsection{Komposition}

Seien $ X,Y,Z $ Mengen, $ f : X \to Y $ und $ g : Y \to Z $. Dann heißt $ g \circ f : X \to Z $ mit $ ( g \circ f )(x) := g( f(x) ) $ für alle $ x \in X $ die Komposition von $ g $ und $ f $.

\begin{bsp}
	Seien $ f : \R \to \R : f(x) = 2x + 3 $ für alle $ x \in \R $ und $ g : \R \to \R : g(x) = x^2 $ für alle $ x \in \R $. Dann ist $ ( f \circ g )(x) = 2x^2 + 3 $ und $ ( g \circ f )(x) = (2x + 3)^2 $.
\end{bsp}

\subsubsection{Identische Abbildung}

Sei $ X $ eine Menge. Dann heißt die Abbildung $ \id_X : X \to X $ mit $ \id_X(x) := x $ für alle $ x \in X $ die identische Abbildung auf $ X $. Man schreibt auch häufig $ \id $, wenn $ X $ nicht angegeben werden muss.

\begin{bem}
	Seien $ X,Y $ Mengen und sei $ f : X \to Y $ bijektiv. Dann gilt
	\begin{itemize}
		\item $ f \circ f^{-1} = \id_Y $,
		\item $ f^{-1} \circ f = \id_X $.
	\end{itemize}
\end{bem} 

\subsubsection{Vereinigung und Durchschnitt einer indexierten Mengenfamilie}

Eine Familie bzw. Schar  $ (A_i)_{i \in I}$ von Teilmengen von $ X $, die durch die Menge $I$ indexiert sind,  ist eine Abbildung $i \mapsto A_i$ von $I$ nach $2^X$. 

Für die Familie $ (A_i)_{i \in I} $ definiert man
\begin{align*}
	\text{den Durchschnitt:} && \bigcap_{i \in I} A_i &:= \{ x \in X : x \in A_i \:\text{für alle}\: i \in I \},\\
	\text{die Vereinigung:} && \bigcup_{i \in I} A_i &:= \{ x \in X : x \in A_i \:\text{für ein}\: i \in I \}.
\end{align*}

\begin{bsp}
	 $A_t := [t-1,t+1]$ ist die Menge aller Punkte in $\R$, deren Entfernung von $t \in \R$ höchstens $1$ ist. Dann ist $\bigcap{t \in [-1,1]} A_t = \{0\}$, denn $0$ ist der einzige Punkt in $\R$, dessen Entfernung zu allen Punkte in $[-1,1]$ höchstens $1$ ist. Des Weiteren ist $\bigcup_{t \in [-1,1]} A_t = [-2,2]$, denn alle Punkte im Intervall $[-2,2]$, und nur diese Punkte auf der reellen Achse $\R$, haben den Abstand höchstens $1$ zu einem Punkt aus $[-1,1]$. 
\end{bsp} 

%\begin{bsp}
%	Sei $ \alpha \in (0,\pi) $ und $ v_0 > 0 $ ($ \nearrow $ Abb. 3). $ K_\alpha $ ist die Flugbahn beim Auswurf eines Objekts mit der Anfangsgeschwindigkeit $ v_0 $ unter dem Winkel $ \alpha $ zu Erde.
%	\begin{gather*}
%		K_\alpha := \{ (x,y) \in R^2 : x = \cos(\alpha)t, y = \sin(\alpha)t - \frac{gt^2}{2}, y \geq 0, t \geq 0 \}\\
%		( K_\alpha )_{\alpha \in (0,\pi)}\\
%		\bigcap_{\alpha \in (0,\pi)} K_\alpha = \{ (0,0) \}\\
%		\bigcup_{\alpha \in (0,\pi)} K_\alpha = \text{alle Werte unter der Parabel ($ \nearrow $ Abb. 4)}
%	\end{gather*}
%\end{bsp}

\subsubsection{Summen und Produkte}

Eine Menge $ X $ heißt endlich, falls $ X = \emptyset $ oder falls eine bijektive Abbildung von $ \is{1}{n} $ nach $ X $ existiert mit $ n \in \N $. Der Wert $ n $ heißt die Anzahl der Elemente (Kardinalität) von $ X $ und wird durch $ |X| $ bezeichnet.

Man setzt die Kardinalität von $ \emptyset $ gleich 0. $ |X| $ ist wohl definiert, d.h. eine Menge kann nicht zwei unterschiedliche Kardinalitäten haben.

Sei $ X $ eine nichtleere endliche Menge. Dann kann $ X $ als $ X = \is{x_1}{x_n} $ dargestellt werden mit  $ x_i \neq x_j \Leftrightarrow i \neq j $ für alle $ i,j \in \is{1}{n} $.

Für eine Abbildung $ f : X \to \R $ definiert man
\begin{align*}
	\sum\limits_{x \in X} f(x) &:= f(x_1) + \ldots + f(x_n),
\\
	\prod\limits_{x \in X} f(x) &:= f(x_1) \cdot \ldots \cdot f(x_n).
\end{align*}

Im Fall $ X = \emptyset $ definiert man für $ f : X \to \R $ und $ \sum\limits_{x \in X} f(x) = 0 $ und $ \prod\limits_{x \in X} f(x) = 1 $. Die Summe und das Produkt über eine Menge $X$ sind wohldefiniert (d.h., die beiden Werte sind von der Nummerierung $x_1,\ldots,x_n$ der Elemente von $X$ unabhängig).

F'ru $n \in \N_0$ stehen die Bezeichnungen $\sum_{i=1}^n$ bzw. $\prod_{i=1}^n$ für die Summe bzw. das Produkt über alle ganzzahligen $i$ mit $1 \le i \le n$. 



\subsection{Prädikate}

\subsubsection{Prädikat}

Sei $ X $ Menge. Dann heißt $ P : X \to \{ \text{falsch},\text{wahr} \} $ \emph{Prädikat} auf $ X $. Etwa $ P : \N \to \{ \text{falsch},\text{wahr} \}, P(k) := $ \textquote{$ k(k+1) $ ist durch 3 teilbar} für alle $ k \in \N $.

Durch ein Prädikat $ P : X \to \{ \text{falsch},\text{wahr} \} $ kann man die Menge $ \{ x \in X : P(x) \} $ definieren.

\subsubsection{Quantoren}

$ \forall\: x \in X : P(x) $ für ein Prädikat $ P $ auf eine Menge $ X $ steht für die Aussage \textquote{die Bedingung $ P(x) $ gilt für alle $ x \in X $.} $ \forall $ heißt das \emph{Allgemeinheitsquantor} (Bedeutung: für $ \forall $lle). \\[10pt]
%
$ \exists\: x \in X : P(x) $ bezeichnet die Aussage \textquote{die Bedingung $ P(x) $ gilt für ein $ x \in X $.} $ \exists $ heißt \emph{Existenzquantor} (Bedeutung: es $ \exists $xistiert).

\begin{bem}
	Negierung von Aussagen:
	\begin{enumerate}
		\item $ \overline{\forall\: x \in X : P(x)} \Leftrightarrow \exists\: x \in X : \overline{P(x)} $
		\item $ \overline{\exists\: x \in X : P(x)} \Leftrightarrow \forall\: x \in X : \overline{P(x)} $
	\end{enumerate}
\end{bem}

\begin{bem}
	$ \forall $ und $ \exists $ lassen sich kombinieren. Etwa, wenn man ein Prädikat $ P $ auf $ X \times Y $ hat ($ X,Y $ Mengen), so kann man die Aussagen $ \left( \forall\: x \in X \:\exists\: y \in Y : P(x,y) \right) $, $ \left( \exists\: x \in X \:\forall\: y \in Y : P(x,y) \right) $ usw. einführen.
\end{bem}

\begin{bsp}
	Sei $ (a_n)_{n \in \N} $ Folge reeller Zahlen (mit anderen Worten: $ a : \N \to \R $) und sei $ \alpha \in \R $. Dann heißt $ \alpha $ Limes von $ (a_n)_{n \in \N} $, falls das Folgende gilt:
	\begin{equation*}
		\forall\: \epsilon \in \R^+ \:\exists\: N \in \N \:\forall\: n \in \N: \left( (n \geq N) \Rightarrow (|a_n - \alpha| < \epsilon) \right)
	\end{equation*}
\end{bsp} 

\clearpage
\subsection{Relationen}

\subsubsection{Relation}

Seien $ X,Y $ Mengen. Dann heißt eine Teilmenge $ R $ von $ X \times Y $ eine (binäre) \emph{Relation} zwischen (den Elementen von) $ X $ und $ Y $.

Wenn für $ x \in X $ und $ y \in Y $ die Bedingung $ (x,y) \in R $ gilt, so schreibt man $ x R y $. Wenn $ X = Y $, dann sagt man, dass $ R $ eine (binäre) Relation auf $ X $ ist.

\begin{bsp}\ % hack to force itemize to new line
\begin{itemize}
	\item $ X $ - Menge von Fahrzeugen\\
	$ Y $ - Menge von Features von Fahrzeugen

	\begin{tabular}{l|c|c|c|c}
		& Ersatzrad & Radio & Navi & Automatik \\
		\hline
		$ f_1 $ & 1 & 1 & 1 & 1 \\
		$ f_2 $ & 1 & 1 & 1 & 0 \\
		$ f_3 $ & 0 & 0 & 1 & 1 \\
		$ f_4 $ & 0 & 1 & 1 & 0
	\end{tabular}
	\item $ \leq, <, \geq, > $ auf $ \R $
	\item $ \subseteq $ als Relation auf $ 2^X $ für eine Menge $ X $
	\item Für $ a,b \in \N $ schreibt man $ a | b $, wenn $ b $ durch $ a $ ohne Rest teilbar ist.
\end{itemize}
\end{bsp}

\subsubsection{Äquivalenzrelation}

Sei $ X $ Menge und $ \sim $ eine Relation auf $ X $. Dann heißt $ \sim $ eine Äquivalenzrelation, falls:
\begin{enumerate}
	\item $ \sim $ ist \emph{reflexiv}, d.h. $ x \sim x $ für alle $ x \in X $.
	\item $ \sim $ ist \emph{symmetrisch}, d.h. $ x \sim y $ ist äquivalent zu $ y \sim x $ für alle $ x \in X $.
	\item $ \sim $ ist \emph{transitiv}, d.h. aus $ x \sim y $ und $ y \sim z $ folgt $ x \sim z $ für alle $ x,y,z \in X $.
\end{enumerate}
Für eine Äquivalenzrelation $ \sim $ auf einer Menge $ X $ und ein $ x \in X $ heißt
\begin{equation*}
	[x]_\sim := \{ y \in X : x \sim y \}
\end{equation*}
die Äquivalenzklasse von $ x $ bzgl. $ \sim $. Die Menge aller Äquivalenzklassen von $ \sim $ ist
\begin{equation*}
	X/{\sim} := \{ [x]_\sim : x \in X \}. % {\sim} supresses space between to "/"
\end{equation*}

\begin{bsp}\
\begin{itemize}
	\item Sei $ V $ endliche Menge und sei $ \binom{V}{2} := \{ \{ u,v \} : u,v \in V, u \neq v \} $. Das Paar $ (V,E) $ mit $ E \subseteq \binom{V}{2} $ heißt \emph{Graph} mit Kantenmenge $ V $ und Knotenmenge $ E $.
	
	$ G = (V,E), G = \{ 1,\ldots,6 \}, E = \{ \{ 1,2 \}, \{ 2,3 \}, \{ 3,4 \}, \{ 4,1 \}, \{ 1,3 \}, \{ 5,6 \} \} $
	
	Für $ a,b \in V $ heißt $ b $ von $ a $ aus \emph{erreichbar} (im Graphen $ G = (V,E) $), falls ein $ k \in \N_0 $ und Elemente $ u_0,\ldots,u_k \in V $ existieren mit $ u_0 = a $, $ u_k = b $ und $ \{ u_i,u_{i+1} \} \in E $ für alle $ i \in \N_0 $ mit $ i < k $.
	
	Die Erreichbarkeit ist eine Äquivalenzklasse auf $ V $.	Die Äquivalenzklassen (Zusammenhangskomponenten) für dieses Beispiel sind $ \{ 1,2,3,4 \} $ und $ \{ 5,6 \} $.
	
	\item Sei $ m \in \N $. Für $ a,b \in \Z $ sagt man, dass $ a $ kongruent zu $ b $ modulo $ m $ ist, falls $ a-b \in m\Z $, wobei $ m\Z := \{ mz : z \in Z \} $.
	
	Schreibweise: $ a \equiv b \mod{m} $.
	
	Die Kongruenz modulo $ m $ ist eine Äquivalenzrelation auf $ \Z $.
	
	\item Sei $ \sim $ Relation auf $ \Z \times \N $, definiert durch $ (a,b) \sim (c,d) $ für $ a,c \in \Z, b,d \in \N $, wenn $ ad = bc $ gilt.
	
	Diese Relation ist eine Äquivalenzrelation (Aufgabe).
	
	D.h. jede rationale Zahl ist eine Äquivalenzklasse von diesem $ \sim $.
\end{itemize}
\end{bsp}



\section{Vollständige Induktion}

\begin{framed} 
Sei $P : \N \to \{ \operatorname{falsch}, \operatorname{wahr} \}$. Dann sind die folgenden Bedingungen äquivalent:
  \begin{itemize}
  	\item[(a)] $P(n)$ gilt für alle $n \in \N$.
  	\item[(b)] $P(1)$ gilt, und für alle $n \in \N$ gilt die Implikation $P(n) \Rightarrow P(n+1)$. 
  \end{itemize} 
\end{framed} 

Wenn man (a) mit Hilfe von (b) zeigt, so sagt man, dass man die \emph{vollständige Induktion} über $n$ benutzt. Die Aussage $P(1)$ nennt man den \emph{Induktionsanfang}, die Annahme, $P(n)$ sei erfüllt, die \emph{Induktionsvoraussetzung}, und die Herleitung von $P(n+1)$ aus $P(n)$ den \emph{Induktionschritt}. 

Wenn man die Äquivalenz von (a) und (b) für das Prädikat $P(1) \wedge \cdots \wedge P(n)$ an der Stelle von $P(n)$ benutzt, so erhält man, dass (a) auch zur folgenden Behauptung äquivalent ist: 

\begin{itemize}
	\item[(c)] $P(1)$ gilt, und für alle $n \in \N$ gilt die Implikation $P(1) \wedge \cdots \wedge P(n) \Rightarrow P(n+1)$. 
\end{itemize} 

Aussage~(c) ist eine Variante der vollständigen Induktion, mit der Induktionsannahme, $P(k)$ sei für alle $k \in \N$ mit $k \le n$ erfüllt. 

Des Weiteren benutzt man naheliegende Varianten der vollständigen Induktion für Aussagen der Form ``$P(n)$ gilt für alle $n \ge n_0$'' für Prädikate $P : \{ n \in \Z \,:\, n \ge n_0\} \to \{ \operatorname{falsch}, \operatorname{wahr} \}$ und $n_0 \in \Z$. Bei der Induktion in diesem Fall ist die Aussage $P(n_0)$ der Induktionsanfang. 

\begin{bsp}
	Wir zeigen den Existenz-Teil des Fundamentalsatzes der Arithmetik: Jede natürliche Zahl ist Produkt endlich vieler Primzahlen.  Etwas formaler heißt das: jedes $n \in \N$ besitzt die Faktorisierung $\prod_{i=1}^k p_i$ mit $k \in \N_0$, wobei alle $p_i$ in dieser Faktorisierung Primzahlen sind. Wir zeigen die Aussage durch Induktion über $n$. Die Aussage gilt für $n=1$ mit $k=0$ ($1$ ist Produkt von $0$ Primzahlen). Sei $n \in \N$ und man nehme an, allen Zahlen aus $\{1,\ldots,n\}$ lassen sich in endlich viele Primfaktoren zerlegen. Wir betrachten nun die Zahl $n+1$. Ist $n+1$ Primzahl, so ist sie Produkt $n+ 1 = \prod_{i=1}^k p_k$ mit $k=1$ und $p_1=n+1$. Wenn $n+1$ keine Primzahl ist, so existieren $a, b \in \N$ mit $a, b \ge 2$ und $n+1 = ab$. Aus $a,b \ge 2$ und $ab = n+1$ folgt, dass $a$ und $b$ in $\{1,\ldots,n\}$ liegen: denn man hat $a = \frac{n+1}{b} \le \frac{n+1}{2} \le n$, und analog auch $b \le n$. Nach der Induktionsvoraussetzung lassen sich $a$ und $b$ in endlich viele Primfaktoren zerlegen: $a = \prod_{i=1}^s q_i$ mit $b = \prod_{j=1}^t r_j$, wobei $s, t\in \N$ und alle $p_i$ und $r_j$ Primzahlen sind. Dann ist $n+1 = ab = q_1 \cdot \ldots \cdot q_s \cdot r_1 \cdot \ldots \cdot r_t$ Faktorisierung von $n+1$ in $s+t \in \N$ Primfaktoren. 
\end{bsp} 

\section{Eliminationsverfahren zur Lösung von LGS}

\subsection{Lineare Gleichungssysteme (LGS)}

Im Folgenden seien $ \K $ ein Körper, $ m,n \in \N $ und $ a_{ij}, b_i \in \K $ für alle $ i \in \is{1}{m}, j \in \is{1}{n} $. Wir definieren ein\emph{lineares Gleichungssystem} bzgl. des Körpers $\K$ als ein System der Form
\begin{equation}
	\sum_{j=1}^{n} a_{ij} x_j = b_i \qquad  (i=1,\ldots,m). 
	\label{B1.1}
\end{equation}
mit unbekannten $ x_1, \ldots, x_n $ aus $ \K $. Die Menge
\begin{equation}
	X := \{(x_1, \ldots, x_n) \in \K^n : \text{\eqref{B1.1} gilt} \}
	\label{B1.2}
\end{equation}
heißt die Lösungsmenge von \eqref{B1.1}.

Das Ziel ist die folgenden Fragen algorithmisch zu beantworten:
\begin{enumerate}
	\item
		Ist $ X = \emptyset $?
	\item
		Besteht X aus genau einem Element aus $ \mathbb{K}^n $?
\end{enumerate}
Außerdem wollen wir in dem Fall $ X \neq \emptyset $ eine explizite Beschreibung von $ X $ ermitteln. Man schreibt \eqref{B1.1} oft als eine Tabelle:
\begin{center}
\begin{tabular}{c|ccc|c}
	& $ x_1 $ & $ \ldots $ & $ x_n $ & \\
	\hline
	$ z_1 $ & $ a_{1,1} $ & $ \ldots $ & $ a_{1,n} $ & $ b_1 $ \\
	$ z_2 $ & $ a_{2,1} $ & $ \ldots $ & $ a_{2,n} $ & $ b_2 $ \\
	\vdots & \vdots &  & \vdots & \vdots \\
	$ z_m $ & $ a_{m,1} $ & $ \ldots $ & $ a_{m,n} $ & $ b_m $ \\
	\hline
\end{tabular}
\end{center}
Hierbei nutzen wir $ z_i $ als eine (optionale) Bezeichnung für die $ i $-te Gleichung ($ z $ wie Zeile). 

\begin{bsp}
	Die Gleichung 
	\[
		 \left\{ \begin{array}{rcrcr}
		x_1 &+& x_2 &=& 15\\
		x_1 &-& 2x_2 &=& 3
	\end{array} \right. 
	\]
	wird tabellarisch als 
	\begin{center}
	\begin{tabular}{r|rr|r}
		& $ x_1 $ & $ x_2 $ & \\
		\hline
		$ z_1 $ & 1 & 1 & 15 \\
		$ z_2 $ & 1 & -2 & 3 \\
		\hline
	\end{tabular}
	\end{center} 
	dargestellt. 
\end{bsp}
 

